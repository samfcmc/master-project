%!TEX root = ../report.tex

% 
% Abstract 
% 

\begin{abstract}
% Proximity-based apps are... POIs
% Increasing popularity
% Based on this the concept of Smart Place is introduced
% We have several technologies
% Bluetooth Low Energy can be used indoors
% The space owner only need to install small devices,
% BLE Beacons that broadcast an identifier
% Apps can get this identifier and offer different
% possibilities to the user according to it
% Development process implies to write the code
% for getting beacons' ids and get information
% about each POI from the backend
% One app need to be installed for each Smart Place.
% Here, we introduce a solution to make development
% of apps for smart places much easier and faster
% Developers have abstractions for the technology
% that is being used and for communication with the backend
% Also, in this solution, they write the code using
% technologies for web development, such as HTML, CSS and Javascript
% as they would do for creating any web application
% The users and owners of Smart Places just need to install
% one app. With this app, users can interact with the
% Smart Place and the owners can manage it in order to choose
% the apps that will run on their places.

Proximity-based applications engage users while they are in proximity
of Points of Interest (POIs). These apps are becoming popular
between the users of mobile devices.
The concept of a Smart Place is defined as a place that 
has some POIs that
allows users to interact with them.
Several technologies can be used.
GPS can be used to locate places outside, where they can pick up
satellite signals. 
Bluetooth Low Energy (BLE) Beacons perform well
indoors. They broadcast an identifier, using
BLE, that mobile
apps can receive and allow them to offer 
different possibilities to the users.
To develop apps for Smart Places we need
to write the code that gets the beacons' identifiers and
get more information about the POI for each identifier from
a backend. 
Also, users need to install one app for each Smart Place.
Here, we propose a solution to make the development of 
mobile
apps for smart places much easier and faster than the
existing
solutions, with benefits for both users and owners of
such Smart Places. Users will only need one app to interact
with any Smart Place. Owners will use the same app to
configure the POIs.

\end{abstract}