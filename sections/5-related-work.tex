%!TEX root = ../report.tex

%
% Related work
%

\section{Related Work}
\label{related_work}
In this section we describe related work about
development and deployment of mobile native and web 
context-aware applications and we discuss the 
limitations and the main benefits of each one of
the following solutions, providing, at the same time,
motivation to develop our framework.
\begin{description}
  \item[Dynamic frameworks for Mobile web DTN applications]
  There are multiple frameworks to deploy localized
  mobile native or web apps, even if the user is using an
  unstable Internet connection. One example of this
  kind of frameworks is the one
  described in \cite{Sankaran2014}.
  This framework supports the deployment of localized 
  native and
  web apps in Delay Tolerant Networks (DTNs).
  A DTN is an approach to computer networks that 
  addresses issues about the lack of continuous network
  connectivity. Mobile networks are good examples of DTNs. 
  It also has an embedded web server 
  and has an implementation that runs on a PC, which allow
  developers to test the applications before deploying
  to production environment.
  However, the owner of a given point of interest does not
  have access to any interface to install apps that will
  belong to that point. Also, it does not support, in the
  same app, multiple points of interest with different
  meanings and different behavior in each one of them. 
  \item[Dynamix]
  Dynamix \cite{Carlson2012} is a framework to develop
  mobile native and web apps that allow them to receive
  context information, for instance, position and device's
  orientation. This framework works with plug-ins that take
  one or more sensor's raw data and turn that into event
  objects, that contain more high-level information.
  This framework supports many kinds of context information
  and it is possible to develop more plug-ins to allow the
  apps to access some context information that is not
  already supported. It is similar to the previous
  framework but is more focused on delivering context
  information to apps instead of deploying them.
  To achieve our goal, our framework could be just a
  plug-in for Dynamix. The plug-in would
  need to get the beacon's raw data and
  turn that into a more high-level information 
  using a back-end. In this framework,
  the user needs to install an app to manage the service
  that runs in background and needs to define some
  security policies. This could imply a big overhead since
  we are more focused on developing proximity-based
  web applications, that don't require such complex security
  policies. In our framework, the user only need to
  install one app, let it run in background and turn on
  bluetooth. Also, this does not offer a complete solution
  since the owner of one or more points of interest
  has no way of installing new apps on them.
  \item[Ionic] Ionic is a framework to develop mobile apps
  using web technologies, such as HTML and JavaScript. The
  same code can be deployed, as a native app, to any mobile
  platform. Using some plug-ins, the apps can access some
  device's functionalities, such as bluetooth. However,
  the developer has to configure a back-end and write the
  code that will interact with it. Also, since the apps are
  installed like native ones, we still have the problem of
  having one app for each smart place.
\end{description}
