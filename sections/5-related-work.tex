%!TEX root = ../report.tex

%
% Related work
%

\section{Related Work}
\label{sec:related_work}
In this section we discuss some related work.
For each one we describe its limitations and its main
benefits, providing, at the same time, motivation
to develop our framework.

\subsection{Proximity-based apps}
\label{sub:ble_beacons_applications}
% Why
% For each one: What, Pros, Cons
BLE Beacons are going to
be used to develop proximity-based apps
on top of it. Some
applications, where this technology is used,
will be presented here, to
get good insight about the potential use cases of this 
technology and the apps developed using it.
\\
% BlueSentinel
\textbf{BlueSentinel\cite{Conte2014}:} is a 
occupancy detection system, for smart buildings,
that uses BLE Beacons to detect the presence of
people. The concept of a smart building
is similar to Smart Place,
due to the existence of sensors and actuators.
It is focused on the power efficiency of the
building. The idea is to optimize energy
consumption according to people's presence.
For instance, if there are no people in a given room,
the heating system can be turned off.
In this solution, the users have to install
an app, that will get the beacons' signal and
send data to a server, which will process it
and send requests to actuators in order to
perform actions to optimize the
building's power efficiency.
Unfortunately, there is a limitation
of iBeacon protocol implementation
in iOS. Beacons can be received, by the apps,
only when these are active. When the apps are in
background, they are waken up only to handle
enter/exit region events. To circumvent this
limitation, the authors developed custom
beacons, which advertise more than one region
in a cyclic sequence. These custom beacons
were created using an 
Arduino\footnote{http://www.arduino.cc/}
and an Bluetooth USB BLE dongle.
Since this solution is a native app,
users have to install it in order
to make the smart building work to
optimize power efficiency.
Once the user starts the app, he does not
need to interact with it anymore, since it
will run in background.
\\
% Blue View
\textbf{BlueView\cite{Chen2013}:} is a system to help
visually impaired people to perceive some POIs.
This solution has two main components: The viewer device
and the Beacon Points (BPs). The first one is a mobile phone,
carried by the user, which is bluetooth-enabled.
The Beacon Points are just bluetooth tags instead of
BLE Beacons. The name of a POI is associated with
MAC address of the tag which it is associated to.
The steps involved in using the system are the
following: first, the viewer device will scan
for nearby BPs; then, a list of the names of
BPs is created. This list is refreshed anytime a new
BP is detected and the user is informed through auditory
feedback. The second step consists of the user, using
the viewer device, establishing a connection with a BP
attached to an object. Finally, using audio prompt, the BP
will assist the user in locating the object.
Despite of this solution being a mobile app, installed
in the viewer device, the authors do not have in
consideration the typical concerns of any mobile app,
such as the energy consumption.
The authors tested the application, in 2013,
using Nokia N70 as the viewer device.
This solution could be implemented using BLE Beacons
and the viewer device could be any Android, iOS or
Windows Phone smartphone.
For the audio features the smartphone's speaker or
a custom BLE Beacon with a built-in speaker could be
used.
\\
% MOSES
\textbf{MOSES
(Mobile Opportunistic System for Experience Sharing)\cite{BenAbdesslem2014}:}
% In exhibitions, share experiences and communicate with
% other participants
% Roaming charges... No Internet access
% They have smartphones with comm techs like bluetooth and wifi
% Share experience (text or multimedia content)
% Shared content is geo-tagged
% Available to participants in the same location
% Or will be visiting the area later on
% Using DTN
% When content is created is shared with other users
% in range in an epidemic fashion
% Each user share his content and content
% received by other users
% Implemented in Android
% Geo-tagging:
% -> GPS, WiFi, cellular positioning
% -> Indoor, estimote beacons
% -> Equipment: Smartphones, Beacons, Raspberry Pies
% Raspberry pie: Collect all content created by participants
% and emulate an intermittent internet connection
% Act as a WiFi access point
is a solution to allow
participants of exhibitions, to share experiences and
communicate with other participants.
The participants can share their experiences using
just text or they can add multimedia content.
This shared content is geo-tagged, which means that
it will be available to other participants in the same
location.
Most of these events are international. 
Due to the
roaming charges most users will not have an
Internet connection available in their mobile devices.
However, most of the participants have smartphones
equipped with communication technologies such as
Bluetooth and WiFi.
To circumvent this limitation, this solution
uses a Delay Tolerant Network (DTN)\cite{pateldelay}.
When content is created it is shared with other
users in range. Each user shares the created content and the content received by other users.
This solution was implemented in Android.
Geo-tagging is performed using usual positioning
systems, such as GPS, WiFi and cellular positioning.
For indoor positioning it uses 
Estimote\footnote{http://estimote.com/} beacons.
Besides the smartphones and the beacons, this
solution also uses a 
Raspberry Pi\footnote{http://www.raspberrypi.org/}.
This device is used to collect all content created by
participants and it emulates an intermittent Internet
connection, acting as a WiFi access point.
\\
% Context capture
\textbf{ContextCapture\cite{Antila2011}:}
% Usage of context-based awareness in status updates
% Allows users to add different descriptions of context 
% information to their Twitter messages
% and Facebook status updates in a narrative format
% Indoor location: BLE Beacons
% Two main goals:
% Demonstrates technical aspects of collaborative
% context
% Test and analyze the UX of context-aware systems
% Include contextual information to status updates
% in Facebook and Twitter 
% User can decide the abstraction level
% (Coordinates, address or semantic label
% such as "conference venue")
% Mobile app and server-side integrated with Facebook and 
% Twitter
% Mobile app gathers context from device itself,
% sensors and from nearby devices using Bluetooth
% Format: “[User-defined message] Sent
% from [Location] while [Activity]
% [Description] [Topic] and [Applications Activity] with 
% [Friends].”
In this work, the authors try to use
context-based information to allow users to
add more information to their status updates
in the main social networks, such as
Facebook\footnote{http://www.facebook.com} and 
Twitter\footnote{http://twitter.com}.
This work had two main goals: first, demonstrate technical
aspects of collaborative context, such as,
how to get contextual information from
surrounding devices and how they can be used
as a source of contextual information;
second, test and analyze the user experience of
context-aware systems.
The user can decide the abstraction level (coordinates,
address or semantic label).
The authors implemented a mobile app and a
server integrated with Facebook and Twitter.
Context information comes from the smartphone itself,
from its sensors and from the nearby devices through
Bluetooth. 
Devices can be other smartphones or BLE Beacons, which
are used for indoor location.
Similarly to \cite{BenAbdesslem2014}, devices communicate
with each other as a network.
Using this solution, the user can create status updates,
in the mentioned social networks, in the following format:
\\
``
[User-defined message] 

Sent from [Location] while [Activity]

[Description] [Topic] and [Applications Activity] with 
[Friends].
''
\\
\subsection{Context-awareness}
\label{sub:frameworks_context_aware}
In this section we describe related work about the
development and deployment of mobile native and web 
context-aware applications.
Since we propose a framework to develop
proximity-based apps, the state
of the art of existing frameworks, that deliver
context information to the apps will be presented.
\\
% DTN
\textbf{Dynamic frameworks for 
mobile web DTN applications:}
There are multiple frameworks to deploy localized
mobile native or web apps, even if the user is using an
unstable Internet connection. One example of this
kind of framework is the one
described in Sankaran et al.\cite{Sankaran2014}.
This framework supports the deployment of localized 
native and
web apps in Delay Tolerant Networks (DTNs).
A DTN is a computer network that 
addresses issues about the lack of continuous network
connectivity. 
Mobile networks are good examples of DTNs,
because, most of the times, the connection is not stable.
Usually, the user is moving, along with his mobile 
device.
It is expected that the connection sometimes is not
available.  
This solution also has an embedded web server 
and has an implementation that runs on a PC, which allow
developers to test their applications before deploying
to production environment.
However, the owner of a given point of interest does not
have access to any interface to install apps that will
belong to that point. Also, it does not support, in the
same app, multiple points of interest with different
meanings and different behavior in each one of them.
\\
% Framework for Developing Distributed Location Based 
% Applications
\textbf{Frameworks for developing distributed
location-based applications:}
% Develop location based apps
% Discuss several technologies for positioning
% GPS, GSM, WiFi and Bluetooth
% Advantages and disadvantages of each one
% Architecture
% -> GPS receiver, bluetooth receiver, WiFi receiver
% -> Mobile device
% -> Database Server, Application server
% Application server provides web services for
% mobile clients and communicates with the
% database server
% Client communication (SOAP message)
% Querying the database (native communication)
% Mobile device get geographical coordinates from
% several devices;
% Send this data, in a SOAP message, to the
% appropriate web service in the Application Server
% App server communicates with Database server
% to query the database
% Database server sends a response back with
% location information if there is any, for that
% particular group of coordinates
% No evaluation
% Location info can come from any source
% Only for native apps (No web apps)
% Does not take into account constraints in terms
% of resources, such as lack of internet connection
% and battery
% SOAP could be replaced by REST [PAPER]
There are frameworks to develop location-based
applications, such as the ones described in
\ref{sub:ble_beacons_applications}, that
try to offer abstractions for the location
technologies.
In the work presented in Krevl et al.\cite{Krevl2006}, 
a framework
was developed to allow developers to build
location-based apps. Location information can come
from any source, such as GPS receivers, Bluetooth
receivers and WiFi receivers.
The authors discuss some benefits and limitations
of several technologies for getting the
user's positioning.
In terms of architecture, the main components
are:
\begin{itemize}
\item
Devices that are used to get location data, such as
the ones already mentioned. 
\item The users' mobile devices.
\item The Database Server, which is where the mapping
between geographical coordinates and location
information is stored.
\item And, the Application Server, which provides web services for
mobile clients. This server also communicates
with the Database Server.
\end{itemize}
The mobile device get geographical coordinates
from any source and send that data, in a
SOAP \cite{Seely:2001:SCP:560836} message,
to the appropriate web service in the Application
Server. This server, communicates with the Database Server
to query the database, which sends back a response with
location information, if there is any, for that
particular group of geographical coordinates.
The authors did not evaluate the system.

This solution offers abstractions for the location
information sources. Geographical coordinates can
come from any source. It is a good approach for
mobile native apps, but, it does not support web apps.
The authors do not take into consideration
constraints in terms of resources, such as
lack of Internet connection and battery.
Since most users have limited data plans for
their smartphones and SOAP messages can
grow, in size, due to its XML format,
a more efficient message encoding could be used
instead.
\\
% Dynamix
\textbf{Dynamix\cite{Carlson2012}:}
is a framework to develop
mobile native and web apps that allow them to receive
context information, for instance, position and device's
orientation. This framework has plug-ins that get
one or more sensor's raw data and turn that into event
objects, that contain more high-level information.
This framework supports many kinds of context information
and it is possible to develop more plug-ins to allow the
apps to generate additional events that are not
already supported. It is similar to the previous
framework but is more focused on delivering context
information to apps instead of deploying them.

To achieve our goal, our framework could be just a
plug-in for Dynamix. The plug-in would
need to get the beacon's raw data and
turn that into a more high-level information 
using a backend. In this framework,
the user needs to install an app, that manages the service
that runs in background, and needs to define some
security policies. This could mean a big overhead since
we are more focused on developing proximity-based
web applications, that do not require such complex security
policies because, in this kind of apps, there is only need
to access the device's sensors that could provide,
to the applications, positioning data. 
In our framework, the user would only need to
install one app, let it run in background and turn on
bluetooth. Also, this does not offer a complete solution
since the owner of one or more points of interest
has no way of installing new apps on them.

\subsection{APIs and Frameworks}
\label{sub:frameworks_web}
Since we propose a framework to develop proximity-based
mobile web apps, a good insight, about some existing
solutions to develop mobile apps using web technologies,
such as HTML, CSS and Javascript, is needed.
These solutions, described below, allow developers to
write their apps, using HTML, CSS and Javascript, as
if they are web apps and deploy them in most used app
stores, such as 
Google Play\footnote{https://play.google.com/}
and
App Store \footnote{http://store.apple.com/}.
\\
\textbf{Cordova\footnote{http://cordova.apache.org/}:}
is a set of APIs that allow developers to access
devices' functionalities, such as the camera or
accelerometer, from Javascript code.
To access such functionalities developers build plugins.
Any developer can develop plugins and use plugins
developed by others.
\\
%Phonegap
\textbf{PhoneGap\footnote{http://phonegap.com/}:}
is a framework to develop mobile apps
using web technologies, such as HTML and JavaScript. The
same code is deployed, as a native app, to multiple mobile
platforms. Each app is nothing more than an embedded browser
that runs a local web application. 
Using some plugins from Cordova, the apps can access some
device's functionalities, such as bluetooth. 
However, to develop proximity-based apps, such as
the ones mentioned in 
\ref{sub:ble_beacons_applications},
the developer has to configure a backend and write the
code that will interact with it. Also, since the apps are
installed as native ones, the user needs to install
an app for each place where each app works.
\\
%Ionic
\textbf{Ionic\footnote{http://ionicframework.com/}:}
is a framework very similar to PhoneGap. When using
this framework, developers are allowed to use Cordova
plugins, similarly to PhoneGap. It includes a
Command-Line Interface to initialize the structure
of a new app. Also, a new app already includes Javascript
files that use 
AngularJS\footnote{https://angularjs.org/}, 
which is a framework, for frontend development,
to build web pages with dynamic content.
However, it has the same benefits and limitations
than PhoneGap.

\subsection{Summary}
\label{sub:summary}
To finish this section, each related work has its own
advantages and limitations.
In \ref{sub:ble_beacons_applications} some
examples of proximity-based apps were described.
In \ref{sub:frameworks_context_aware} there is
some discussion about some
frameworks for developing context-aware mobile apps,
including their limitations.
Most of them, only support mobile native apps and
do not offer a complete solution, since the developer,
needs to configure a backend and write the code that
receive the beacons' signals and sends data retrieved by 
the beacons to the backend, in order to build 
proximity-based apps using BLE Beacons.
Finally, in \ref{sub:frameworks_web}, there are
examples of frameworks for developing mobile apps
using HTML, CSS and Javascript, allowing developers
to write the code as they are creating web applications
and deploy them in the app store of more than one
mobile Operating System.
In terms of development of proximity-based apps,
these solutions have the same disadvantages of the
ones in \ref{sub:frameworks_context_aware}.
None of those solutions offer abstractions to
the technology used to get information about POIs.
Our framework, described in the next section, tries 
to offer a complete solution that solves all
these limitations, in such a way that, developers do not
need to think about the technology behind proximity-based
apps, but only about the application logic and the meaning
of each point of interest.