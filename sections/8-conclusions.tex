%!TEX root = ../report.tex

%
% Conclusions
%

\section{Conclusions}
\label{sec:conclusions}
% A Smart Place is...
% Several technologies can be used to develop them
% GPS does not work well indoors
% BLE Beacons are being used for this kind of apps
% (examples in related work)
% There are already solutions to develop context-aware mobile apps
% They can be used to develop apps for smart places
% Write code to get beacons' identifiers and get POIs
% for those identifiers from the backend
% We will try to create abstractions for the beacons and
% for the backend
% Developers would only need to think about POIs.
% We are going to evaluate the solution, taking into
% account constraints in terms of resources
% This solution intends to make the development
% of apps for smart places much easier and faster
% Also, users will not need to install one app
% for each smart place. Only one app will be needed
% in order to allow the user to interact with any
% smart place.
A Smart Place is a set of POIs that the user can interact
with using a mobile device, such as a smartphone or a tablet.
Several technologies can be used to develop apps for
Smart Places. GPS does not require the place owner to
install any hardware. However, as mentioned in 
section \ref{sec:introduction}, it does not work well
indoors. BLE Beacons are becoming a solution to
develop this kind of apps. They use Bluetooth,
which most smartphones have. Also, because they use
Bluetooth Low Energy, they do not consume much energy.
There are already solutions to develop apps for these
places, where we have multiple POIs and we want to
offer different features, to the user, according
to nearby POIs. However, developers need
to write very similar code, in all apps, to
get the beacons' identifiers and get the information
about the corresponding POIs from a backend.
Also, the apps are native ones.
The solution proposed here tries to create abstractions
for the BLE Beacons and for the backend, allowing
developers to spend most of their time writing the
code for the application logic as they should,
instead of in the details of the technology that is
being used. Besides the benefits for developers,
the main benefit for the users is that they only need
to install one single app to interact with any Smart Place.
Since it is a solution for mobile devices, an evaluation
that takes into account constraints in terms of resources,
such as Internet connection and energy, will be performed.
This work has the main goal of making the development
of this kind of apps much easier and faster than
the already existing solutions. Also, the need of only
one app, for the users, will provide to them, an alternative
to QR Codes, with the main difference there is no need of
any scan of any code, also allowing more kinds of interactions
instead of just opening an web page. 
