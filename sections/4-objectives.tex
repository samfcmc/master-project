%!TEX root = ../report.tex

% 
% Objectives
% 

\section{Objectives}
\label{sec:objectives}
% ~1 page
%Clearly explain the project objectives.

% Develop a framework that allows...
% Develop an app for a smart place, abstracting
% beacons, think only in POI and their meaning
% Users only need one app for any smart place
% The same app can be used, for space owners, to install apps
We propose
to build a framework to develop proximity-based web mobile
apps, or, according to our definition of Smart Place, web
mobile apps for Smart Places.
Part of the framework will need to run in the user's 
mobile device.
The other part will run on the device's browser and
it will be what developers will have access to.
One of the main goals is to implement the framework itself.
Also, a Smart Place will be created and 
two versions of a simple
app for it will be developed, using our approach and using a SDK to interact with the beacons.
We want to test how much our solution improves the
development of mobile apps for Smart Places.
In this project, we want to create a solution that
gives, to developers, abstractions for the technology
of the POIs and the backend.
Also, we want to make these Smart Places accessible by 
anyone who has a mobile device with Bluetooth, at least,
version 4.0 and without the need to install a large number
of apps.
Another goal we want to achieve is to allow owners of
Smart Places to configure the POIs using just one app.
All they would need to do is just to install the BLE Beacons
that they want, in their Smart Places, and use the app
to choose which app will ``run'' there and what is the meaning
of each POI in that app.


% Given that, we have two main objectives. First, the framework
% itself and second, the Smart Places App.

% The framework should allow developers to develop their apps,
% without needing to write the code to get the beacons' signals
% and get more information from the backend. That would be the
% framework's job. Developers would only need to write the code
% that will run in each POI. Each POI would have a name and
% some parameters specified by the developer. 
% For instance, in the example of the restaurant,
% in \ref{sec:introduction}, the POI could have the name Table and
% a parameter could be the number of the table.
% Developers would only need to write their apps using web
% technologies, such as HTML and JavaScript.

% Part of the framework will run on the users' smartphones.
% We will develop the Smart Places App, for Android, which would
% be an  app that will scan for beacons, and will get the 
% corresponding information about the POIs that are being 
% represented by the scanned beacons.
% The user will be notified about the POIs
% that were found and then, he can select one of them and
% start using the associated app. 
% Also, the smart place owner, would be able to use this app
% to install apps, developed using this framework, in his
% smart place.

% To summarize, the framework will allow developers to develop
% apps for smart places using nothing more than web technologies
% and without writing the code to scan nearby beacons.
% The Smart Places App will allow the users to access any app,
% developed using this framework, in any smart place. Also,
% the smart places owner could install apps in their spaces
% using this same app.
