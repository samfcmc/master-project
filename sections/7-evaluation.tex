%!TEX root = ../report.tex

% 
% Evaluation
% 

\section{Evaluation}
\label{sec:evaluation}
% 1/2 pages
%Explain how you are going to show your results (statistical 
% data, cpu performance etc). Answer the following questions:
%\begin{itemize}
 % \item Why is this solution going to be better than others.
 % \item How am I going to defend that it is better.
%\end{itemize}

% What to measure: Data consumption, memory usage, CPU usage
% Considering usage of a cache (see architecture)
% Why measure this... Reference CMov Article
% Experiments conditions... Same smart place, same app
% Concrete use case (which app)
% Compare cache vs no cache...
% Compare using framework vs not using framework

To evaluate the proposed solution we need to 
take into account its
architecture. As mentioned in section 
\ref{sec:architecture},
the component that receives, as input, the beacon raw data and
returns, as output, the information about a POI,
which is the
Client, can get that data from the Backend or 
from the Cache.
The use of this Cache is optional as it is just a 
way to reduce
the communication between the app and the Backend.
Since our solution will run in a mobile environment, 
which is a resource constrained environment,
despite of some of the smartphones today having
good CPUs and 1GB or more of RAM.
% Add reference to this! (see introduction)
Table \ref{tab:smartphones} shows specifications about some 
of the most recent smartphone models.
% Specifications about high end smartphones
\begin{table}[h]
\centering
\begin{tabular}{|c|c|c|c|c|c|}
\hline
\textbf{Smartphone} & \textbf{\begin{tabular}[c]{@{}c@{}}Launch \\ date\end{tabular}} & \textbf{\begin{tabular}[c]{@{}c@{}}Operating \\ System\end{tabular}} & \textbf{CPU} & \textbf{RAM} & \textbf{\begin{tabular}[c]{@{}c@{}}Internal\\ Storage\end{tabular}} \\ \hline
\begin{tabular}[c]{@{}c@{}}Samsung \\ Galaxy S5\end{tabular} & \begin{tabular}[c]{@{}c@{}}February, \\ 2014\end{tabular} & Android & \begin{tabular}[c]{@{}c@{}}Quad-core\\ 2.5 Ghz\end{tabular} & 2 GB & \begin{tabular}[c]{@{}c@{}}16 or\\ 32 GB\end{tabular} \\ \hline
\begin{tabular}[c]{@{}c@{}}LG \\ Nexus 5\end{tabular} & \begin{tabular}[c]{@{}c@{}}October, \\ 2013\end{tabular} & Android & \begin{tabular}[c]{@{}c@{}}Quad-core\\ 2.3 Ghz\end{tabular} & 2 GB & \begin{tabular}[c]{@{}c@{}}16 or\\ 32 GB\end{tabular} \\ \hline
\begin{tabular}[c]{@{}c@{}}Apple \\ iPhone 6\end{tabular} & \begin{tabular}[c]{@{}c@{}}September, \\ 2014\end{tabular} & iOS 8 & \begin{tabular}[c]{@{}c@{}}Dual-core\\ 1.4 Ghz\end{tabular} & 1 GB & \begin{tabular}[c]{@{}c@{}}16, 64 or\\ 128 GB\end{tabular} \\ \hline
\begin{tabular}[c]{@{}c@{}}Nokia \\ Lumia 920\end{tabular} & \begin{tabular}[c]{@{}c@{}}September,\\ 2012\end{tabular} & \begin{tabular}[c]{@{}c@{}}Windows\\ Phone 8\end{tabular} & \begin{tabular}[c]{@{}c@{}}Dual-core\\ 1.2 Ghz\end{tabular} & 1 GB & 32 GB \\ \hline
\end{tabular}
\caption{Specifications about some smartphones}
\label{tab:smartphones}
\end{table}
There is a need to consider energy consumption and other resources'
usage, such as, memory, CPU and transfered data through the
Internet connection.

We need to define a use case, a practical implementation
of an app for a smart place to run the experiments
to take such measurements. We will put three beacons,
each one in a different room and the app will show, to the user,
some information about the points where the beacons
are installed.
This app should be implemented using our solution.
However, we need to compare ours to the existing
alternatives. To be able to perform such comparison
we will implement this app in two ways.
First, using our solution and second without using
our solution, which means that it will be a native
app that will use some SDK to interact with the
beacons.

Some experiments will be performed in order to compare
our approach to the classic one, which is to develop a
native app that uses some SDK to interact with the beacons
and more code to get information from a backend.
Table \ref{tab:experiments} summarizes these experiments.
For each one, there are some conditions specified, such as,
the data connection type, which can be WiFi or 3G, if
it is using the Cache from our architecture (explained in 
\ref{sec:architecture}), if the mobile app that is being
used was implemented using this solution or if it is
a native app using an SDK. 
Also, the metrics and the number of beacons
are the same for all experiments.

% Please add the following required packages to your document preamble:
% \usepackage{multirow}
\begin{table}[h]
\centering
\begin{tabular}{|c|c|c|c|c|c|c|}
\hline
\multirow{2}{*}{\textbf{}} & \multicolumn{6}{c|}{\textbf{Experiments}} \\ \cline{2-7} 
 & \textbf{1} & \textbf{2} & \textbf{3} & \textbf{4} & \textbf{5} & \textbf{6} \\ \hline
\textbf{\begin{tabular}[c]{@{}c@{}}Data Connection \\ Type\end{tabular}} & WiFI & 3G & WiFi & 3G & WiFi & 3G \\ \hline
\textbf{Using Cache} & -- & -- & No & No & Yes & Yes \\ \hline
\textbf{\begin{tabular}[c]{@{}c@{}}Using our solution /\\ Using a native SDK\end{tabular}} & SDK & SDK & \begin{tabular}[c]{@{}c@{}}Our\\ solution\end{tabular} & \begin{tabular}[c]{@{}c@{}}Our\\ solution\end{tabular} & \begin{tabular}[c]{@{}c@{}}Our\\ solution\end{tabular} & \begin{tabular}[c]{@{}c@{}}Our\\ solution\end{tabular} \\ \hline
\textbf{Metrics} & \multicolumn{6}{c|}{CPU usage, Memory usage and data consumption} \\ \hline
\textbf{\# of Beacons} & \multicolumn{6}{c|}{3} \\ \hline
\end{tabular}
\caption{Summary of experiments that will be performed in evaluation}
\label{tab:experiments}
\end{table}

These experiments, as mentioned below, 
will measure resources' usage over time, 
and compare both implementations. In each
experiment using our solution, we will
compare both approaches, using the Cache
and not using it.

We will not measure power consumption directly.
Instead we will measure Input/Output (I/O)
operations, because as shown in 
\cite{Pathak2012} they are interrelated.
Comparing these
two types of data connections could provide us
some information about the
overhead implied by the communication with
the backend. Also, it would be useful to
know if it could be used using
a usually, limited data plan, provided by
the operator.

