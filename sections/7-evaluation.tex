%!TEX root = ../report.tex

% 
% Evaluation
% 

\section{Evaluation}
\label{sec:evaluation}
% 1/2 pages
%Explain how you are going to show your results (statistical 
% data, cpu performance etc). Answer the following questions:
%\begin{itemize}
 % \item Why is this solution going to be better than others.
 % \item How am I going to defend that it is better.
%\end{itemize}

% What to measure: Data consumption, memory usage, CPU usage
% Considering usage of a cache (see architecture)
% Why measure this... Reference CMov Article
% Experiments conditions... Same smart place, same app
% Concrete use case (which app)
% Compare cache vs no cache...
% Compare using framework vs not using framework

To evaluate our solution we need to take into account its
architecture. As mentioned in section \ref{sec:architecture},
the component that receives, as input, the beacon raw data and
returns, as output, the information about a POI, which is the
Client, can get that data from the Back-end or from the Cache.
The use of this Cache is optional. It is just a way to reduce
the communication between the app and the Back-end.
Since our solution will run in a mobile environment, 
which is a resource constrained environment,
despite of some of the smartphones today have
good CPUs and 1GB or more of RAM.
% Add reference to this! (see introduction)
Table \ref{tab:smartphones} shows specifications about some 
of the most recent smartphone models.
% Specifications about high end smartphones
\begin{table}[h]
\centering
\begin{tabular}{|l|l|l|l|l|l|l}
\cline{1-6}
Smartphone & Launch date & Operating System & CPU & RAM & Internal storage
\\
\cline{1-6}
Samsung Galaxy S5 & February 2014 & Android & Quad-core 2.5 Ghz & 2 GB & 16 or 32 GB
\\
\cline{1-6}
LG Nexus 5 & October 2013 & Android & Quad-core 2.3 Ghz & 2 GB & 16 or 32 GB
\\
\cline{1-6}
Apple iPhone 6 & September 2014 & iOS 8 & Dual-core 1.4 Ghz & 1 GB & 16, 64 or 128 GB
\\
\cline{1-6}
Nokia Lumia 920 & September 2012 & Windows Phone 8 & Dual-core 1.5 Ghz & 1GB & 32 GB
\\
\hline
\end{tabular}
\caption{Specifications about some smartphones}
\label{tab:smartphones}
\end{table}
We need to consider energy consumption and other resources
usage, such as, memory, CPU and transfered data through the
Internet connection.

In our evaluation we will measure memory and CPU usage
and data transfered between the app and the back-end.
We can take conclusions about battery consumption since
more the CPU, memory and communications usage,
more the battery consumption.

We need to define a use case, a practical implementation
of an app for a smart place to run the experiments
to take such measurements. We will put some beacons
in a building and the app will show, to the user,
some information about the points where the beacons
are installed.
This app should be implemented using our solution.
However, we need to compare ours to the existing
alternatives. To be able to perform such comparison
we will implement this app in two ways.
First, using our solution and second without using
our solution, which means that it will be a native
app that will use some SDK to interact with the
beacons.

This experiments, as mentioned below, 
will measure resources' usage over time, 
and compare both implementations. In each
experiment using our solution, we will
compare both approaches, using the Cache
and not using it. Also, each experiment
will run twice, for the two most used
data connections, WiFi and 3G.
Experiments, such the one
in \cite{Pathak2012}, shows that most of the energy, 
in mobile apps,
is spent in I/O operations.
Comparing these
two types of data connections could provide us
some information about the
overhead implied by the communication with
the back-end. Also, it would be useful to
know if it could be used using
a usually, limited data plan, provided by
the operator.

